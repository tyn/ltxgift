\documentclass[dvipdfmx,20pt]{beamer}
\usetheme{Madrid}
\usepackage{ltxgift}
\title{クイズ教材の例}
\giftdef{\ZZ}{\mathbf{Z}}
\def\CorrectMark#1{#1(正)}
\def\WrongMark#1{#1(誤)}
\begin{document}

\Category{デモ}

\begin{gift}
\begin{frame}{問題(\QTitle{既約多項式})}
\Question{
  \(\ZZ_2\) 上の多項式 \( x^2+x+1 \) は既約である。
}
\begin{enumerate}
\item \Correct{Yes}
\item \Wrong{No}
\end{enumerate}
\end{frame}
\end{gift}

\begin{gift}
\begin{frame}{問題}
\Question{
  情報ビット系列の直後にパリティ検査ビット系列が並ぶ
  符号をなんと呼ぶか。
}
\begin{enumerate}
\item \Wrong{コンパクト符号}
\item \Correct{組織的符号}
\item \Wrong{巡回符号}
\end{enumerate}
\end{frame}
\end{gift}

\begin{gift}
\begin{frame}{問題(\QTitle{元の位数})}
\Question{
  \(\alpha\) は多項式\(x^2+x+1 \in \ZZ_2[x]\)の根とする。
  \(\alpha\) の位数はどれか。
}
\begin{enumerate}
\item \Wrong{2}\Response{→ 位数の復習が必要です}
\item \Correct{3}
\end{enumerate}
\end{frame}
\end{gift}

\begin{gift}
\begin{frame}{問題(\QTitle{無限積分})}
\Question{
  次の積分の値を求めなさい。
  \[ \int_{-\infty}^\infty e^{-x^2}dx \]
}
\begin{enumerate}
\item \Wrong{$\pi$}
      \Response{→二乗して極座標表示を使います。}
\item \Correct{$\sqrt{\pi}$}
\item \Wrong{$\frac{\sqrt{\pi}}{2}$}
      \Response{→変数変換時の積分区間を再確認しましょう。}
\item \Wrong{$\frac{\pi}{2}$}
      \Response{→二乗して極座標表示を使います。}
\end{enumerate}
\end{frame}
\end{gift}
\end{document}
